\section{Glossario}
Qui sono racchiuse tutte le definizioni e gli acronimi che sono stati  utilizzati nella documentazione.
\subsection{Definizioni}

\paragraph{Scalabilità} si riferisce alla capacità di un sistema di aumentare o diminuire le risorse a seconda delle necessità.

\paragraph{Usabilità} l'usabilità di un prodotto è il grado con cui esso può essere usato da specificati utenti per
raggiungere specificati obiettivi con efficacia, efficienza e soddisfazione in uno specificato
contesto d'uso.
\paragraph{Mock-up} Un mock-up, è una realizzazione a scopo illustrativo o meramente espositivo di un oggetto o un sistema, senza le complete funzioni dell'originale. Un mockup può rappresentare la totalità o solo una parte dell'originale di riferimento (già esistente o in fase di progetto), essere in scala reale oppure variata.

\paragraph{Class diagram} I class diagram (diagrammi delle classi, in italiano) sono uno dei tipi di diagrammi che possono comparire in un modello UML.
\\
In termini generali, consentono di descrivere tipi di entità, con le loro caratteristiche e le eventuali relazioni fra questi tipi. Gli strumenti concettuali utilizzati sono il concetto di classe del paradigma object-oriented e altri correlati.
\paragraph{Sequenc diagram} Un Sequence Diagram (Diagramma di sequenza, in italiano) è un diagramma previsto dall'UML utilizzato per descrivere uno scenario.

\paragraph{Scenario} Uno scenario è una determinata sequenza di azioni in cui tutte le scelte sono state già effettuate; in pratica nel diagramma non compaiono scelte, né flussi alternativi.

\paragraph{State chart diagram} Uno state chart (diagramma di stato, in italiano) è un tipo di diagramma usato in informatica per descrivere il comportamento dei sistemi, il quale viene analizzato e rappresentato tramite una serie di eventi che potrebbero accadere per ciascuno stato. Per poter essere realizzato, il sistema deve essere composto da un numero finito di stati.

\paragraph{Activity diagram} Un activity diagram (diagramma di attività in italiano) è un tipo di diagramma che permette di descrivere un processo attraverso dei grafi in cui i nodi rappresentano le attività e gli archi l'ordine con cui vengono eseguite.

\paragraph{Persona} Una persona in UX design è un identikit vero e proprio di un utente (fittizio) ideale, che viene utilizzato per identificare una fascia d'utenti, i quali esprimono le loro esigenze, comportamenti ed interessi.  

\paragraph{White-Box} Modalità di testing in cui un metodo viene testato in base al codice che contiene.

\paragraph{Black-Box} Modalità di testing in cui un'unità viene testata in base ai requisiti.

\paragraph{Framework} \`{E} un'architettura logica di supporto sul quale un software può essere progettato e realizzato.

\paragraph{Back-end} Rappresenta la parte che permette l'effettivo funzionamento di queste interazioni.

\paragraph{Front-end} Rappresenta la parte che interagisce con l'utente finale.

\paragraph{Android} Sistema operativo realizzato da Google.

\paragraph{Server}  Programma che ha il compito di offrire servizi ai client richiedenti.

\paragraph{Tabella Cockburn} Tabelle utilizzate per la rappresentazione dei casi d'uso.

\paragraph{Use case} Rappresenta una funzione o servizio offerto dal sistema ad uno o più attori.

\paragraph{Docker} \`{E} un popolare software libero progettato per eseguire processi informatici in ambienti isolabili, minimali e facilmente distribuibili chiamati container Linux, con l'obiettivo di semplificare i processi di deployment di applicazioni software.
\newpage
\subsection{Acronimi}
\paragraph{AWS} Amazon Web Services è un'azienda statunitense di proprietà del gruppo Amazon, che fornisce servizi di cloud computing su un'omonima piattaforma \textit{on demand}.

\paragraph{EC2} Uno dei servizi di cloud offerti da AWS.

\paragraph{API} Un API (Application Program Interface), in un programma informatico, in italiano "interfaccia di programmazione dell'applicazione", si indica un insieme di procedure atte a risolvere uno specifico problema di comunicazione tra diversi computer, tra diversi software, tra diversi componenti di software.
Spesso tale termine designa le librerie software di un linguaggio di programmazione.

\paragraph{HTTP} Sta per HyperText Transfer Protocol, è un protocollo a livello applicativo usato come principale sistema per la trasimissione di informazioni via web.

\paragraph{ISO/IEC} ISO e IEC cooperano strettamente per creare uno standard internazionale sviluppato specificatamente per la gestione dei servizi IT.

\paragraph{GDPR} General Data Protection Regulation, è un regolamento, sviluppato dalla comunità europea, che disciplina il trattamento dei dati personali da parte di aziende.

\paragraph{MVP} Model View Presenter è un design pattern, principalmente utilizzato per software Android.

\paragraph{UI} User Interface è il termine inglese usato per indicare l'interfaccia grafica.

\paragraph{DTO} Data Transfer Object è un design pattern per trasferire dati da una parte back-end a quella front-end (ove necessario).

\paragraph{SECT} \`{E} una specifica modalità di testing. Consiste nell'effettuare i test tramite prodotto cartesiano dei parametri.

\paragraph{WECT} \`{E} una specifica modalità di testing. Consiste nell'effettuare il minimo numero di test case che ricoprono tutte le classi d'equivalenza valide e uno per quelle non valide.