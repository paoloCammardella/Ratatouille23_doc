\section{Introduzione}
Ratatouille23\texttrademark\ è un software sviluppato e progettato dal gruppo INGSW2223\_N\_46 per conto della società SoftEngUniNA\texttrademark. \newline
Gli sviluppatori si prendono carico della verifica dei moduli software necessari al corretto funzionamento di tale sistema.
Ratatouille23\texttrademark\ è un software che offre il supporto alla gestione di attività di
ristorazione.\newline
Il sistema consiste in un’applicazione performante e affidabile, attraverso la quale gli utenti
possono fruire delle funzionalità del sistema in modo intuitivo, rapido e piacevole ed una parte back-end che garantisce la comunicazione veloce ed affidabile tra i client.

\subsection{Cosa contiene questa documentazione}
Questa documentazione contiene tutte le informazioni necessarie a comprendere il funzionamento del software Ratatouille23\texttrademark
\begin{itemize}
    \item Documento dei Requisiti Software.
    \item Documento di Design del sistema.
    \item Definizione di un piano di testing e valutazione sul campo dell’usabilità.
\end{itemize}

\subsection{Tecnologie utilizzate}
Lo sviluppo del software Ratatouille23\texttrademark \ è avvenuto tramite:
\begin{itemize}
    \item \textbf{Client}:
          \begin{itemize}
              \item Android con linguaggio object oriented (Java).
              \item Retrofit: richieste HTTP lato android.
          \end{itemize}
    \item \textbf{Server}:
          \begin{itemize}
              \item Framework Java Spring boot.
              \item RestTemplate.
          \end{itemize}
    \item \textbf{Database}: PostgreSQL.
    \item \textbf{Testing}: Suite JUnit.
\end{itemize}
Il tutto è affiancato tecnologie allo stato dell'arte quali \textit{Docker} per separare in \textit{container} back-end e database, e servizi di \textit{Cloud Computing} come \textit{AWS},
al fine di massimizzare la scalabilità del sistema in vista di un possibile repentino aumento del numero degli utenti nelle fasi
iniziali di rilascio al pubblico.